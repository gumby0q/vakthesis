%%
%% This is file `xampl-bib.tex',
%% generated with the docstrip utility.
%%
%% The original source files were:
%%
%% vakthesis.dtx  (with options: `xampl-bib')
%% 
%% IMPORTANT NOTICE:
%% 
%% For the copyright see the source file.
%% 
%% Any modified versions of this file must be renamed
%% with new filenames distinct from xampl-bib.tex.
%% 
%% For distribution of the original source see the terms
%% for copying and modification in the file vakthesis.dtx.
%% 
%% This generated file may be distributed as long as the
%% original source files, as listed above, are part of the
%% same distribution. (The sources need not necessarily be
%% in the same archive or directory.)
%% xampl-bib.tex  Приклад файла-оболонки для списку/списків літератури
% Рядки, що починаються з %GATHER, призначені для WinEdt
% Якщо є лише один список літератури, оточення bibset не потрібно використовувати
%GATHER{xampl-thesis.bib}
\begin{bibset}{Список використаних джерел}
\bibliographystyle{gost2008}
% Для сортування літератури за алфавітом використовуйте
%\bibliographystyle{gost2008s}
\bibliography{xampl-thesis}
\end{bibset}
%GATHER{xampl-mybib.bib}
\begin{bibset}[a]{Список публікацій автора}
\bibliographystyle{gost2008}
\bibliography{xampl-mybib}
\end{bibset}
