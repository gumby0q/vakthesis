%%
%% This is file `xampl-intro.tex',
%% generated with the docstrip utility.
%%
%% The original source files were:
%%
%% vakthesis.dtx  (with options: `xampl-intro')
%% 
%% IMPORTANT NOTICE:
%% 
%% For the copyright see the source file.
%% 
%% Any modified versions of this file must be renamed
%% with new filenames distinct from xampl-intro.tex.
%% 
%% For distribution of the original source see the terms
%% for copying and modification in the file vakthesis.dtx.
%% 
%% This generated file may be distributed as long as the
%% original source files, as listed above, are part of the
%% same distribution. (The sources need not necessarily be
%% in the same archive or directory.)
%% xampl-intro.tex  Приклад вступу до дисертації
% Приклад ненумерованого розділу
\chapter*{Вступ}


\paragraph{Актуальність теми}

Це не є справжня дисертація. Це лише приклад, який повинен
допомогти користувачу підготувати свій файл. Але я зробив його із
своєї дисертації. Тому формули, теореми, доведення, імена, книги і
статті у списку літератури інколи можуть бути справжніми (хоча
можуть здаватися безглуздими, бо вирвані з контексту).


\paragraph{Зв'язок роботи з науковими програмами, планами, темами}

Робота виконана у рамках досліджень математичних об'єктів зі
складною локальною будовою, що проводяться на кафедрі вищої
математики Національного педагогічного університету імені
М.",П.",Драгоманова.


\paragraph{Мета і завдання дослідження}

Метою роботи є розробка основ метричної теорії дійсних чисел,
представлених рядами Остроградського $1$-го виду, та застосування
отриманих результатів до дослідження математичних об'єктів зі
складною локальною будовою (фрактальних множин, сингулярних та
недиференційовних функцій, сингулярно неперервних мір).

\subparagraph{Методи дослідження}

У роботі використовувалися методи математичного аналізу, теорії
функцій дійсної змінної, теорії міри, метричної теорії чисел,
теорії ймовірностей, фрактального аналізу тощо.


\paragraph{Наукова новизна одержаних результатів}

Основними науковими результатами, що виносяться на захист, є такі:
% Приклад ненумерованого списку
\begin{itemize}
\item Доведено, що множина неповних сум ряду Остроградського
$1$-го виду є ніде не щільною досконалою множиною нульової міри
Лебега та нульової розмірності Хаусдорфа--Безиковича.

\item Знайдено умови нуль-мірності (додатності міри) певних класів
замкнених ніде не щільних множин чисел, заданих умовами на
елементи їх розвинення в ряд Остроградського $1$-го виду.

\item \ldots
\end{itemize}


\paragraph{Практичне значення одержаних результатів}

Робота має теоретичний характер. Отримані результати є безперечним
внеском у теорію міри, метричну теорію чисел, теорію функцій
дійсної змінної та теорію сингулярних розподілів ймовірностей.
Запропоновані в дисертації методи можуть бути корисними при
дослідженні математичних об'єктів зі складною локальною будовою,
заданих за допомогою інших представлень чисел з нескінченним
алфавітом, зокрема рядів Остроградського $2$-го виду.


\paragraph{Особистий внесок здобувача}

Основні результати, що виносяться на захист, отримані автором
самостійно. Зі статей, опублікованих у співавторстві, до
дисертації включені лише ті результати, що належать автору.


\paragraph{Апробація результатів дисертації}

Основні результати дослідження доповідалися на наукових
конференціях різного рівня та наукових семінарах. Це такі
конференції:
\begin{itemize}
\item Український математичний конгрес, Київ, 21--23 серпня
2001~р.;

\item \ldots
\end{itemize}
Це такі семінари:
\begin{itemize}
\item семінар відділу теорії функцій Інституту математики НАН
України (керівник: чл.-кор. НАН України О.",І.",Степанець);

\item \ldots
\end{itemize}


\paragraph{Публікації}

Основні результати роботи викладено у 6~статтях
% Тут не наводяться всі статті. Це лише приклад
\cite{Bar98fasp1,Bar98fasp2}, опублікованих у виданнях, що внесені
до переліку наукових фахових видань України, та додатково
відображено в матеріалах конференцій~\cite{PrB01umc}.


\paragraph{Зміст роботи}

Тут викладають основні результати дисертації. Це, напевно, зручно
для потенційного читача, для опонентів. Наявність чи відсутність
цього пункту залежить від традицій школи. ВАК не рекомендує і не
забороняє такий пункт у вступі дисертації.

Далі йде безглуздий текст. Не читайте його. Тут немає нічого
розумного (чи хоча б цікавого), оскільки не передбачалося, що
хтось це читатиме. Просто необхідно трохи тексту, щоб сторінку
чимось заповнити. Вважайте, що це щось на кшталт \emph{Lorem
ipsum}. Крім того, за допомогою цієї сторінки з безглуздим текстом
можна порахувати кількість рядків на сторінці та символів у рядку.

Здається, у мене закінчуються запаси безглуздого тексту. Хто б міг
подумати, що так складно писати текст лише для заповнення
сторінки! Він, крім того, ще й неефективний, оскільки не всі букви
можна тут побачити. Але деякі гарні букви і цифри можна
роздивитися: а, б, в,~\ldots, 1, 2, 3,~\ldots, а ще такі: \emph{а,
б, в,~\ldots}.

Досить! Далі йде осмислений (я сподіваюся) текст. Він наведений
тут зовсім не для того, щоб порушити  права Юрія Андруховича чи
видавництва <<Фоліо>>. Просто у мене під руками був електронний
варіант <<Таємниці>>. Чому Рябчук був абсолютним ґуру для них
усіх?

<<По-перше, він у всьому був жахливо переконливий, у всьому "---
як у своїх статтях, так і в розмовах. З ним було безнадійно
полемізувати, його слід було тільки слухати. Йому було на той час
29 років, тобто він був ще й фізично старший від решти товариства,
а це в тому віці суттєво, ця різниця між 29 і, скажімо, 22. Це не
те що 45 і 38 "--- там уже фактично жодної різниці немає. А між 90
і 83 "--- й поготів. Так от, він був старший і досвідченіший, з
ним можна було про все на світі радитися, бо він на той час уже
змінив з десяток різних занять, був тричі одружений і розлучений,
жив самотнім даосом у запущеній старій віллі на Майорівці, з таким
же запущеним старим садом і не менш запущеними старими сусідами.
Він був цілком прозорий від аскетизму (інший тут сказав би, що він
\emph{аж світився}), щодня стояв на голові і правильно, згідно з
Ученням, дихав, а харчувався виключно пісним рисом без солі. На
той час його статті про літературу вже почали публікувати і він
потроху ставав авторитетом не тільки в андеґраунді. Ага, м'ятний
чай "--- він пив багато м'ятного чаю. Його двокімнатне помешкання
у тій віллі являло собою досить інтенсивну суміш з усяких
речей-уламків, але передусім воно було захаращене книгами,
газетами і рукописами. Книги починалися від порогу і ніде не
закінчувалися. Він тримав їх навіть у холодильнику. Якби не книги,
то він і не знав би, на біса йому той холодильник здався. Кажуть,
наче там-таки, у холодильнику, він тримав пришпиленим до задньої
стінки вирізаний з газети портрет Брежнєва. Це називалося
\emph{малим Сибіром}. Ще пару років тому його помешкання стало
такою собі міні-комуною, притулком для тодішніх нефорів: Морозов,
Лишега, Чемодан, Кактус. Останнього я ніколи в житті не бачив, до
сьогодні. Але знаю, що такий був, уявляєш? Здається, саме він
намалював на стіні того сквоту нев'їбенно притягальну фреску з
усіма згаданими особами "--- вони сидять, розпатлані й неголені,
як апостоли, а на столі в них червоне вино і рибина. Не пам'ятаю,
чи мали німби, але припускаю, що цілком могли мати. Це житло стало
такою собі рукавичкою. Кожен із них зносив до Рябчукової хати
всякий непотріб "--- в залежності від того, чим на ту хвилину
перебивався і що звідки вдавалося потягти. Пам'ятаю табличку
ЕКСПОНАТ НА РЕСТАВРАЦІЇ. Ще пам'ятаю ПАЛАТА ДЛЯ НЕДОНОШЕНИХ "--- з
усього випливало, що свого часу котрийсь із гостей цього притулку
підзаробляв на життя пологами. Тепер тобі зрозуміло, чому Микола
Рябчук був абсолютним ґуру?>>


\paragraph{Подяка}

Якщо хочете комусь подякувати, пишіть тут.
