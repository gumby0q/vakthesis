%% mon2017dev-aref.tex  Приклад автореферату дисертації (для mon2017dev-aref.cls)

% Це просто файл-приклад, який дисертантки і дисертанти можуть використати
% як основу для автореферату своєї дисертації.
% Але в цьому базовому вигляді він може не підходити для всіх дисертацій.
% Треба наповнити його конкретним змістом,
% а також зробити певні налаштування згідно зі своїми потребами:
% вибрати потрібні опції класу,
% підключити корисні додаткові пакунки,
% налаштувати параметри сторінки (зокрема розміри берегів),
% означити специфічні команди, які використовуються в дисертації, тощо.

\documentclass[
  % type=phd,% тип дисертації
    %%%%%% Ступені згідно з Порядком присудження наукових ступенів (Постанова КМ)
    % c    кандидат <галузь науки> (типово)
    % d    доктор <галузь науки>
    %%%%%% Ступені згідно із Законом України «Про вищу освіту»
    % phd  доктор філософії
    % artd доктор мистецтва
    % scd  доктор наук
  % instnameorder=asc,% порядок розміщення на титульному аркуші
                      % назви установи і назви органу, до сфери управління якого належить установа
    % desc низхідний (типово)
    % asc  висхідний
    % ascd висхідний порядок з відмінюванням
  % councilsec=true,% шифр спеціалізованої вченої ради в рядку про вченого секретаря (типово false)
  % guide=mon2017dev.naiau,% модуль для підтримки специфічних вимог
    % karazin Харківський національний університет імені В. Н. Каразіна
    %         -- галузь науки для типу дисертації c і d
    %         -- дрібні зміни в оформленні
    % knu     Київський національний університет імені Тараса Шевченка
    %         -- повтор однакових установ двічі
    % naiau   Національна академія внутрішніх справ
    %         -- лапки, тире точно, як у зразку з наказу МОН
    %         -- дрібні зміни в оформленні
    % --------------------------------------------------------------------------
    % kivi    у стилі Киви
    %         -- дисертація за здоров'я наукового рівня
    %            dissertation to the health of the science level
]{mon2017dev-aref}[2021/07/21]
% Для друкування сторінки A5 на аркуші паперу формату A4 доцільно
% використовувати опцію oneside
%\documentclass[a4paper,oneside]{mon2017dev-aref}

% Налагодження кодування шрифта, кодування вхідного файла
% та вибір необхідних мов
\usepackage[T2A]{fontenc}
\usepackage[cp1251]{inputenc}
\usepackage[english,russian,ukrainian]{babel}

% Підключення необхідних пакетів. Наприклад,
% Пакети AMS для підтримки математики, теорем, спеціальних шрифтів
\usepackage[intlimits]{amsmath}
\allowdisplaybreaks
\usepackage{amsthm}
\usepackage{amssymb}
% Налагодження нумерованих списків
%\usepackage{enumerate}
% Гіпертекстові документи
%\usepackage{hyperref}
% або лише спеціальне форматування URL
%\usepackage{url}
% У списку літератури зворотні вказівки на посилання
%\usepackage{backref}
% Сортування посилань
%\usepackage[noadjust]{cite}
% Останні два пакети несумісні між собою. Крім того, конфліктують з цим класом!
% Таблиці зі стовпчиками, що розтягуються
%\usepackage{tabularx}
% Включення факсимільних підписів і взагалі робота з ілюстраціями, кольором тощо
% \usepackage{graphicx}

% Налагодження параметрів сторінки (зокрема берегів).
% Наприклад, за допомогою пакета geometry
% УВАГА!
% Параметри сторінки не зафіксовані жорстко вимогами до оформлення дисертацій,
% але можуть залежати від технічних особливостей конкретної друкарні.
% У цьому прикладі вказано параметри згідно з технічними вимогами 2007 року
% у друкарні Інституту математики НАН України.
% Варто попередньо обговорити і узгодити це питання зі своєю друкарнею.
\usepackage{geometry}
\geometry{total={11cm,17cm},includehead}
% Щоб "розтягнути" сторінку на аркуш формату A4, відкрийте цей рядок
% (але викликати опцію класу a5paper!)
%\geometry{mag=1414}

% Означення теорем (теоремоподібних структур)
% Класичний варіант: для кожної теореми свій лічильник,
% тобто теорема 1.1, лема 1.1, теорема 1.2
\theoremstyle{plain}
\newtheorem{theorem}{Теорема}[chapter]
\newtheorem{lemma}{Лема}[chapter]
\newtheorem{corollary}{Наслідок}[chapter]
\theoremstyle{definition}
\newtheorem{definition}{Означення}[chapter]
\newtheorem{example}{Приклад}[chapter]
\theoremstyle{remark}
\newtheorem{remark}{Зауваження}[chapter]
% Цікавий варіант: всі теореми нумеруються одним лічильником,
% тобто теорема 1.1, лема 1.2, теорема 1.3
%\theoremstyle{plain}
%\newtheorem{theorem}{Теорема}[chapter]
%\newtheorem{lemma}[theorem]{Лема}
%\newtheorem{corollary}[theorem]{Наслідок}
%\theoremstyle{definition}
%\newtheorem{definition}[theorem]{Означення}
%\newtheorem{example}[theorem]{Приклад}
%\theoremstyle{remark}
%\newtheorem{remark}[theorem]{Зауваження}

% Локальні означення
\newcommand{\N}{\mathbb{N}}
\newcommand{\Z}{\mathbb{Z}}
\newcommand{\Q}{\mathbb{Q}}
\newcommand{\R}{\mathbb{R}}
\newcommand{\set}[1]{\left\{#1\right\}}
\newcommand{\abs}[1]{\left\lvert#1\right\rvert}
\newcommand{\norm}[2][]{\left\lVert#2\right\rVert_{#1}}
% Це потрібно для скороченого запису об'єктів, пов'язаних з рядами Остроградського
\newcommand{\Osign}[1]{\mathrm{O}^{#1}}
\newcommand{\bOsign}[1]{\bar{\mathrm{O}}^{#1}}
\makeatletter
\newcommand{\Cset}[2][\bOsign1]{C[#1,
  \if\relax\expandafter\@gobble#2\relax #2\else\{#2\}\fi]}
\makeatother

% \AtBeginDocument{%
%   \def\secretaryname{Вчений секретар\\
%     спеціалізованої вченої ради}%
% }

% Інформація про використані пакети тощо.
% Може знадобитися для відлагодження класу документа
%\listfiles

\begin{document}

% Назва дисертації
\title(uk){Формування професійної іншомовної компетентності майбутніх філологів
  засобами~мультимедійних технологій}
\title(en){Formation of future philologists' professional foreign competence
  by the means of multimedia}
\title(ru){Формирование профессиональной иноязычной компетентности будущих филологов
  посредством мультимедийных технологий}

% Прізвище, ім'я, по батькові здобувача
\author(uk){Сорокіна Наталія Володимирівна}
\author(en){Sorokina Nataliya Volodymyrivna}
\author(ru){Сорокина Наталия Владимировна}

% Факсимільний підпис автора у файлі sorokina-sig.pdf, .eps, .jpeg тощо
% (зсув по x, зсув по y)
% [параметри команди \includegraphics]
% \facsimilesig{author}(0,-30)[width=70pt]{sorokina-sig}

% Прізвище, ім'я, по батькові наукового керівника/консультанта
\supervisor(uk){Редько Валерій Григорович}
% Науковий ступінь, вчене звання наукового керівника/консультанта
               {кандидат педагогічних наук, доцент}
% Установа, де працює науковий керівник/консультант, і посада
               {Інститут педагогіки НАПН України,
                завідувач відділу навчання іноземних мов}
% \supervisor(en){Redko Valeriy Hryhorovych}
%                {candidate of pedagogical sciences, associate professor}
%                {}

% Спеціальність
% Якщо код спеціальності присутній у CSV-файлі, то клас читає інформацію з нього.
% Потрібний CSV-файл вибирається залежно від типу дисертації та основної мови документа.
% За потреби можна підключити інший CSV-файл з шифрами і назвами спеціальностей
% або у факультативному аргументі команди задати всі дані.
% Для типу дисертації c або d
% можна використовувати обидві команди \specialitysci і \specialityedu одночасно,
% щоб отримати наче «змішаний» ступінь:
% тоді на титульному аркуші й в анотації
% буде написано шифри й назви спеціальностей за обома переліками,
% а для типу дисертації c ще й ступінь у вигляді
% «кандидата ... наук (доктора філософії)»

% Варіант для дисертацій на здобуття наукового ступеня кандидата чи доктора ... наук
% (згідно з Переліком наукових спеціальностей)
% Цю команду треба використовувати для типу дисертації c або d.
\specialitysci(uk)[
  % specialityname=Теорія і методика професійної освіти,% спеціальність
  % degreefield=педагогічні,                            % галузь науки, за якою присуджується науковий ступінь
  % specialityfile=<filename>.csv
]{13.00.04}                                             % шифр спеціальності
\specialitysci(en)[
  % specialityname=Theory and methodology of professional education,
  % degreefield=pedagogical,
  % specialityfile=<filename>.csv
]{13.00.04}
\specialitysci(ru)[
  specialityname=Теория и методика профессионального образования,
  degreefield=педагогические,
  % specialityfile=<filename>.csv
]{13.00.04}

% Варіант для дисертацій на здобуття наукового ступеня доктора філософії, доктора мистецтва чи доктора наук
% (згідно з Переліком галузей знань і спеціальностей, за якими здійснюється підготовка здобувачів вищої освіти
% Цю команду треба використовувати для типу дисертації phd, artd або scd,
% але також можна для типу дисертації c або d разом з командою \specialitysci.
% \specialityedu(uk)[
%   % specialityname=Дошкільна освіта,% найменування спеціальності
%   % fieldcode=01,                   % шифр галузі
%   % fieldname=Освіта/Педагогіка,    % галузь знань
%   % specialityfile=<filename>.csv
% ]{012}                              % код спеціальності
% \specialityedu(en)[
%   % specialityname=Preschool education,
%   % fieldcode=01,
%   % fieldname=Education/Pedagogics,
%   % specialityfile=<filename>.csv
% ]{012}
% \specialityedu(ru)[
%   specialityname=Дошкольное образование,
%   fieldcode=01,
%   fieldname=Образование/Педагогика,
%   % specialityfile=<filename>.csv
% ]{012}

% Звичайно, доступна команда \speciality, як у попередніх версіях.
% Клас означує її як одну з команд \specialitysci чи \specialityedu
% залежно від типу дисертації.
% Тому можна використовувати її у такій формі, наприклад:
% \speciality(uk)[degreefield={фізико"=математичні}]{01.01.01}
% \speciality(en){01.01.01}
% \speciality(ru){01.01.01}

% Індекс за УДК
\udc{371.15:81'243}

% Установа, де виконана робота (з вказанням відомчої підпорядкованості)
\institution(uk)[
  % altname=Інститут педагогіки НАПН України,% альтернативна назва установи (пишеться в анотації в описі дисертації)
  % department=кафедра філософії освіти,     % підрозділ установи
  % preposition=на                           % відповідний прийменник: у відділі, на кафедрі, в Інституті тощо (типово: в)
]{Інститут педагогіки, Національна академія педагогічних наук України}
 {Київ}
\institution(en)[
  % altname=Institute of Pedagogics of the NAPS of Ukraine,
  % department=Philosophy of Education Department,
  % preposition=at
]{Institute of Pedagogics, National Academy of Pedagogical Sciences of Ukraine}
 {Kyiv}
\institution(ru)[
  % altname=Институт педагогики НАПН Украины,
]{Институт педагогики, Национальная академия педагогических наук Украины}
 {Киев}

% Прізвище, ім'я, по батькові першого опонента
\opponent{Карташова Любов Андріївна}
% Науковий ступінь, вчене звання першого опонента
  {доктор педагогічних наук, професор}
% Установа, де працює перший опонент, і посада
  {Інститут професійно-технічної освіти Національної академії педагогічних наук України,
   професор кафедри інформаційних технологій}
% Інформація про другого опонента
\opponent{Халимон Ірина Йосипівна}
  {кандидат педагогічних наук, доцент}
  {Ніжинський державний університет імені Миколи Гоголя,
   доцент кафедри прикладної лінгвістики}
% Інформація про третього опонента (для докторських дисертацій)
% ...

% Провідна установа. Відкрийте ці рядки, якщо потрібно
%\linstitution{Національний технічний університет України \linebreak <<КПІ>>,
%  кафедра математичного аналізу та теорії ймовірностей,
%  Міністерство освіти і науки України}
%  {м.~Київ}

% Команду \council переписано в стилі команди \institution:
% ключ institution задає «стандартну» назву установи, у спеціалізованій вченій раді якої проводиться захист дисертації
% (з вказуванням назви органу, до сфери управління якого належить заклад, установа),
% а ключ altname — «альтернативну» (тобто скорочену, для анотації).
% Якщо факультативний аргумент відсутній, то клас вважає,
% що захист проводиться в тій самій установі, де здійснювалася підготовка здобувача,
% а отже, немає потреби писати назву цієї установи двічі на титульному і в анотації.
% Але за потреби можна вручну повторити тут назву установи, і буде повтор на титульному і в анотації.
\council(uk)[
  institution={Національний педагогічний університет імені~М.~П.~Драгоманова,
    Міністерство освіти і науки України},
  altname=Національний педагогічний університет імені~М.~П.~Драгоманова,
  address={01601, м.~Київ, вул.~Пирогова, 9},
  % town=Київ
]{Д~26.053.01}
\council(en)[
  institution={National Pedagogical Dragomanov University,
    Ministry of Education and Science of Ukraine},
  altname=National Pedagogical Dragomanov University,
  % town=Kyiv
]{D~26.053.01}
\council(ru)[
  institution={Национальный педагогический университет имени~М.~П.~Драгоманова,
    Министерство образования и науки Украины},
  altname=Национальный педагогический университет имени~М.~П.~Драгоманова,
  % town=Киев
]{Д~26.053.01}

% Учений секретар ради
\secretary(uk)[
  % degree={кандидат педагогічних наук, доцент},
]{Сиротюк~В.",Д.}
% \secretary(en)[
%   % degree={Candidate of Pedagogical Sciences,\\ Associate Professor},
% ]{Syrotiuk~V.\,D.}

% Факсимільний підпис ученого секретаря у файлі syrotiuk-sig.pdf, .eps, .jpeg тощо
% (зсув по x, зсув по y)
% [параметри команди \includegraphics]
% \facsimilesig{secretary}(-25,-10)[width=70pt]{syrotiuk-sig}

% NOTE: Тут пишеться «інститут{}», щоб запобігти відмінюванню слова «інститут».
% Пакунок casus орієнтується по ключових словах, що треба відмінювати.
% У цій специфічній назві два таких слова: «бібліотека» й «інститут»,
% тому він помиляється.
% \library(uk){Науково-технічна бібліотека імені Г.~І.~Денисенка
%   Національного технічного університету України
%   <<Київський політехнічний інститут{} імені Ігоря Сікорського>>}
%   {03056, м.~Київ, просп.~Перемоги, 37Л}
% \library(en){Hryhoriy Denysenko Library of Science and Technology,
%   National Technical University of Ukraine
%   ``Igor Sikorsky Kyiv Polytechnic Institute''}
%   {03056, Kyiv, 37L Peremohy Ave.}

% Дата захисту і дата розсилання автореферату
% Відкрийте, коли готуєте варіант для ВАК уже після захисту
%\defencedate{2016/05/13}{12:00}
%\postdate{2016/04/12}

% Тут буде обкладинка
\maketitle


% Автореферат не має розділів, підрозділів і т. д., а лише
% структурні частини, що позначаються командою \part
\part{Загальна характеристика роботи}

% \paragraph{Актуальність теми}
\paragraph{Обґрунтування вибору теми дослідження}

Це не є справжній автореферат дисертації. Це лише приклад, який
повинен допомогти користувачу підготувати свій файл. Але я зробив
його із свого автореферату. Тому формули, теореми, доведення,
імена, книги і статті у списку літератури інколи можуть бути
справжніми (хоча можуть здаватися безглуздими, бо вирвані з
контексту).

Оскільки ця частина автореферату майже повторює вступ до
дисертації, то немає сенсу тут наводити якийсь текст, лише назви
пунктів.

% \paragraph{Зв'язок роботи з науковими програмами, планами, темами}
\paragraph{Зв'язок роботи з науковими програмами, планами, темами, грантами}

\paragraph{Мета і завдання дослідження}

% \subparagraph{Методи дослідження}
\paragraph{Методи дослідження}

% \paragraph{Наукова новизна одержаних результатів}
\paragraph{Наукова новизна отриманих результатів}

% \paragraph{Практичне значення одержаних результатів}
\paragraph{Практичне значення отриманих результатів}

\paragraph{Особистий внесок здобувача}

% \paragraph{Апробація результатів дисертації}
\paragraph{Апробація матеріалів дисертації}

\paragraph{Публікації}

Щоб згенерувати список літератури за допомогою Bib\TeX, у тексті
необхідно мати посилання~\cite{Bar98fasp1,Bar98fasp2,PrB01umc}.

% \paragraph{Структура дисертації}
\paragraph{Структура та обсяг дисертації}

Робота складається зі вступу, чотирьох розділів, висновків. Обсяг
дисертації 138~сторінок машинописного тексту, список використаних
джерел (121~найменування) та список публікацій автора
(23~найменування) займають 17~сторінок.


\part{Основний зміст роботи}

В \textbf{розділі~1} вводиться поняття ряду Остроградського $1$-го
виду, елементів та підхідних чисел ряду Остроградського $1$-го
виду\ldots

Далі наведена цитата з <<Порядку присудження наукових ступенів і
присвоєння вченого звання старшого наукового співробітника>>,
затвердженого постановою Кабінету Міністрів України від 07.03.2007
№~423. Лише для того, щоб чимось заповнити сторінку. За допомогою
цієї сторінки можна порахувати кількість рядків на сторінці та
символів у рядку.

<<11. Дисертація на здобуття наукового ступеня є кваліфікаційною
науковою працею, виконаною особисто здобувачем у вигляді
спеціально підготовленого рукопису або опублікованої монографії.
Підготовлена до захисту дисертація повинна містити висунуті
здобувачем науково обґрунтовані теоретичні або експериментальні
результати, наукові положення, а також характеризуватися єдністю
змісту і свідчити про особистий внесок здобувача в науку.

Дисертація, що має прикладне значення, додатково до основного
тексту повинна містити відомості та документи, що підтверджують
практичне використання отриманих здобувачем результатів "---
впровадження у виробництво, достатню дослідно-виробничу перевірку,
отримання нових кількісних і якісних показників, суттєві переваги
запропонованих технологій, зразків продукції, матеріалів тощо, а
дисертація, що має теоретичне значення, "--- рекомендації щодо
використання наукових висновків.

Теми дисертацій пов'язуються, як правило, з напрямами основних
науково-дослідних робіт вищих навчальних закладів або наукових
установ і затверджуються вченими (науково-технічними) радами для
кожного здобувача окремо з одночасним призначенням наукового
консультанта в разі підготовки докторської чи наукового керівника
в разі підготовки кандидатської дисертації.

Мови у дисертації використовуються згідно із законодавством.

12. Дисертація на здобуття наукового ступеня доктора наук є
кваліфікаційною науковою працею, обсяг основного тексту якої
становить 11--13, а для суспільних і гуманітарних наук "--- 15--17
авторських аркушів, оформлених відповідно до державного стандарту.

Докторська дисертація:

повинна містити наукові положення та науково обґрунтовані
результати у певній галузі науки, що розв'язують важливу наукову
або науково-прикладну проблему і щодо яких здобувач є суб'єктом
авторського права;

може бути подана до захисту за однією або двома спеціальностями
однієї галузі науки і повинна відповідати за кожною спеціальністю
вимогам, зазначеним в абзаці третьому цього пункту.

У разі коли дисертація виконана за двома спеціальностями, а
спеціалізована вчена рада, до якої подана дисертація, має право
проводити захист дисертацій лише за однією з них, то за
відсутності в Україні спеціалізованих вчених рад з правом
проведення захисту дисертацій за такими двома спеціальностями з
дозволу ВАК може проводитися разовий захист. Порядок формування
складу спеціалізованої вченої ради для проведення разового захисту
встановлює ВАК.

Наукові положення і результати, які виносилися на захист у
кандидатській дисертації здобувача наукового ступеня доктора наук,
не можуть повторно виноситися на захист у його докторській
дисертації. Ці положення і результати можуть бути наведені лише в
оглядовій частині докторської дисертації.

13. Дисертація на здобуття наукового ступеня кандидата наук є
кваліфікаційною науковою працею, обсяг основного тексту якої
становить 4,5--7, а для суспільних і гуманітарних наук "--- 6,5--9
авторських аркушів, оформлених відповідно до державного стандарту.

Кандидатська дисертація:

повинна містити нові науково обґрунтовані результати проведених
здобувачем досліджень, які розв'язують конкретне наукове завдання,
що має істотне значення для певної галузі науки;

подається до захисту лише за однією спеціальністю.

14. Основні наукові результати дисертації повинні відображати
особистий внесок здобувача в їх досягнення та обов'язково бути
опубліковані ним у формі наукових монографій, посібників (для
дисертацій з педагогічних наук) чи статей у наукових (зокрема
електронних) фахових виданнях України або інших держав. Перелік
наукових фахових видань України затверджує ВАК.

До опублікованих праць, які додатково відображають наукові
результати дисертації, належать дипломи на відкриття; патенти і
авторські свідоцтва на винаходи, державні стандарти, промислові
зразки, алгоритми та програми, що пройшли експертизу на новизну;
рукописи праць, депонованих в установах державної системи
науково-технічної інформації та анотованих у наукових журналах;
брошури, препринти; технологічні частини проектів на будівництво,
розширення, реконструкцію та технічне переоснащення підприємств;
інформаційні карти на нові матеріали, що внесені до державного
банку даних; друковані тези, доповіді та інші матеріали наукових
конференцій, конгресів, симпозіумів, семінарів, шкіл тощо.

Повноту викладу матеріалів дисертації в опублікованих працях
здобувача визначає спеціалізована вчена рада.

Мінімальну кількість та обсяг публікацій, які розкривають основний
зміст дисертацій, визначає ВАК.

Апробація матеріалів дисертації на наукових конференціях,
конгресах, симпозіумах, семінарах, школах тощо обов'язкова.

15. Докторська і кандидатська дисертації супроводжуються окремими
авторефератами обсягом відповідно 1,3--1,9 і 0,7--0,9 авторського
аркуша, які подаються державною мовою. Вимоги до оформлення
автореферату встановлює ВАК.

Автореферат дисертації видається друкарським способом з
обов'язковим зазначенням вихідних відомостей видання у кількості,
визначеній спеціалізованою вченою радою, і надсилається членам
спеціалізованої вченої ради та заінтересованим організаціям не
пізніше ніж за місяць до захисту дисертації. Список адресатів
визначає спеціалізована вчена рада, яка прийняла до захисту
дисертацію. Перелік установ та організацій, яким обов'язково
надсилається автореферат, визначає ВАК.>>


\part{Висновки}

Ряди Остроградського $1$-го виду дозволяють розширити можливості
формального задання і аналітичного дослідження фрактальних множин,
сингулярних мір, недиференційовних функцій та інших об'єктів зі
складною локальною будовою\ldots


% Список опублікованих праць за темою дисертації буде тут. Команда
% \part "захована" всередині команди \bibliography
\bibliographystyle{gost2008}
% Якщо не треба нумерація з крапкою, можна закоментувати наступні три рядки.
\makeatletter
\renewcommand\@biblabel[1]{#1.}
\makeatother
\bibliography{xampl-mybib}


\begin{abstract}[
  language=ukrainian,% мова анотації
  chapter=Анотації,  % заголовок розділу або false, щоб не робити заголовок (типово false)
  % header=false     % автоматична генерація опису дисертації (типово true)
]
Дисертація присвячена дослідженню математичних об'єктів зі
\linebreak складною локальною будовою: фрактальних множин,
сингулярних мір, недиференційовних функцій, заданих у термінах
рядів Остроградського $1$-го виду. Досліджуються деякі класи
замкнених ніде не щільних множин, заданих умовами на елементи їх
розвинення в ряд Остроградського. Встановлено умови нуль-мірності
та додатності міри Лебега множин з цих класів. Проводиться
порівняння з відповідними твердженнями про міру Лебега множин
чисел, заданих умовами на елементи їх розвинення в ланцюговий
дріб. Вказано на принципові відмінності метричної теорії рядів
Остроградського та метричної теорії ланцюгових дробів. Також
вивчено тополого-метричні та фрактальні властивості множини
неповних сум та випадкової неповної суми ряду Остроградського 1-го
виду. Для випадкової величини з незалежними різницями елементів
ряду Остроградського знайдено критерій дискретності
(неперервності) розподілу та умови сингулярності канторівського
типу. Вивчено диференціальні та фрактальні властивості функції,
заданої перетворювачем елементів ряду Остроградського в двійкові
цифри.

\keywords{ряд Остроградського 1-го виду, неповна сума
ряду, міра Лебега, сингулярна міра канторівського типу,
недиференційовна функція, розмірність Хаусдорфа--Безиковича,
фрактал, перетворення Фур'є--Стілтьєса}
\end{abstract}

\begin{abstract}[
  language=russian,
  % chapter=true,
  % header=false
]
Диссертация посвящена исследованию математических объектов со
сложной локальной структурой: фрактальных множеств, сингулярных
мер, недифференцируемых функций, которые заданы в терминах рядов
Остроградского $1$-го вида.

Исследуется множество $\Cset{V_n}$, являющееся с точностью до
счетного множества множеством всех иррациональных чисел,
$\bOsign1$\nobreakdash-\hspace{0pt}символы (т.е. разности
элементов ряда Остроградского) которых удовлетворяют условию
\(g_n(x)\in V_n\subseteq\N\) для всех $n\in\N$. Найдены условия на
множества $V_n$, при которых $\Cset{V_n}$ является множеством
меры~$0$ или множеством положительной меры Лебега. Сделано
сравнение с соответствующими утверждениями о мере Лебега множеств
чисел, заданных условиями на элементы их разложения в цепную
дробь. Указаны принципиальные отличия между метрической теорией
рядов Остроградского 1-го вида и метрической теорией цепных
дробей. В~частности, доказано существование множеств типа
$\Cset{V_n}$ положительной меры таких, что соответствующие
множества, определенные в терминах элементов цепной дроби, имеют
нулевую меру.

Сумма $s=s(\{a_k\})$ ряда
\[
\sum_{k=1}^\infty \frac{(-1)^{k-1}a_k}{q_1q_2\dots q_k}, \quad
\text{где $a_k\in\set{0,1}$},
\]
называется неполной суммой заданного ряда Остроградского $1$-го
вида. Изучены тополого"=метрические и фрактальные свойства
множества неполных сумм ряда Остроградского 1-го вида. Также
изучена структура, тополого"=метрические и фрактальные свойства
случайной неполной суммы ряда Остроградского 1-го вида
\[
\psi=\sum_{k=1}^\infty \frac{(-1)^{k-1}\varepsilon_k}{q_1q_2\dots
q_k},
\]
где $\{\varepsilon_k\}$ "--- последовательность независимых
случайных величин, которые принимают значения $0$ и $1$ с
вероятностями $p_{0k}$ и $p_{1k}$ соответственно.

Для случайной величины
\[
\xi = \sum_{k=1}^\infty
\frac{(-1)^{k-1}}{\eta_1(\eta_1+\eta_2)\dots(\eta_1+\eta_2+\dots+\eta_k)},
\]
$\bOsign1$-символы $\eta_k$ которой являются независимыми
случайными величинами, принимающими значения $1$, $2$,~$\dots$,
$m$,~$\dots$ с вероятностями $p_{1k}$, $p_{2k}$,~$\dots$,
$p_{mk}$,~$\dots$ соответственно, найдено выражение функции
распределения и ее производной, описан спектр распределения.
Найдены критерий дискретности (непрерывности) распределения и
условия сингулярности канторовского типа.

Изучены дифференциальные и фрактальные свойства функции, заданной
следующим образом. Пусть иррациональное число $x$ задано своим
рядом Остроградского с элементами $q_n(x)$, тогда значение функции
\(y=\rho(x)\) определяется двоичной дробью, цифры которой
вычисляются по правилу
\[
\alpha_1=1-\delta_{q_1(x)}, \quad
\alpha_{n+1}=1-\alpha_n\delta_{q_{n+1}(x)},
\]
где $\delta_m$ "--- индикатор множества чётных чисел. В
рациональных точках функция определяется равенством
$\rho(x)=0{,}\alpha_1\alpha_2\dots \alpha_n111\dots$, где
$\alpha_i$ определяются по тому же правилу. Доказано, что эта
функция непрерывна в иррациональных точках, разрывна в
рациональных и нигде не дифференцируема. Множество значений
функции является самоподобным фрактальным множеством.

\keywords{ряд Остроградского 1-го вида, неполная
сумма ряда, мера Лебега, сингулярная мера канторовского типа,
недифференцируемая функция, рaзмeрнoсть Хаусдорфа--Безиковича,
фрактал, преобразование Фурье--Стилтьеса}
\end{abstract}

\begin{abstract}[
  language=english,
  % chapter=true,
  % header=false
]
The thesis is devoted to the investigation of mathematical objects
(fractal sets, singular measures, nondifferentiable functions)
with complicated local structure defined in terms of the first
Ostrogradsky series. We investigate some classes of closed nowhere
dense sets defined by conditions on elements of their expansion in
the Ostrogradsky series. We found conditions for these sets to be
of zero resp.\ positive Lebesgue measure. We compare these results
with the corresponding ones in terms of continued fractions. We
stress the fundamental difference between the metric theory of the
Ostrogradsky series and the metric theory of continued fractions.
We also study topological, metric and fractal properties of the
set of incomplete sums of the Ostrogradsky series and of the
random incomplete sum of the Ostrogradsky series. For random
variables with independent differences of elements of the
Ostrogradsky series we found the criterion for discreteness and
conditions for Cantor-type singularity. We study the differential
and fractal properties of the function defined by the transduser
of elements of the Ostrogradsky series to binary digits.

\keywords{the first Ostrogradsky series (Pierce
expansion), incomplete sum of series, the Lebesgue measure,
Cantor-type singular measure, nondifferentiable function,
Hausdorff--Besicovitch dimension, fractal, Fourier--Stieltjes
transform}
\end{abstract}


% Випускні дані друкарні
% Відкрити наступні два рядки, якщо потрібно на окремому аркуші
%\newpage
%\pagestyle{empty}
\parindent0pt

\hbox{}
\vfill

\hrulefill

% Це приклад з друкарні Інституту математики НАН України
% Тут користувач вписує "свої" дату, к-сть друкованих аркушів,
% (можливо тираж), номер замовлення і адресу друкарні
Підписано до друку 20.03.2007. Формат $60\times84/16$. Папір друк.
Офсет. друк. Фіз. друк. арк. 1,5. Умовн. друк. арк. 1,4.

Тираж 100 пр. Зам. 89.

\hrulefill

Інститут математики НАН України,

01601, м.~Київ-4, вул. Терещенківська, 3.

\end{document}
